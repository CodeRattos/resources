\documentclass[journal,12pt,twocolumn]{IEEEtran}
%
\usepackage{setspace}
\usepackage{gensymb}
\usepackage{xcolor}
\usepackage{caption}
%\usepackage{subcaption}
%\doublespacing
\singlespacing

%\usepackage{graphicx}
%\usepackage{amssymb}
%\usepackage{relsize}
\usepackage[cmex10]{amsmath}
\usepackage{mathtools}
%\usepackage{amsthm}
%\interdisplaylinepenalty=2500
%\savesymbol{iint}
%\usepackage{txfonts}
%\restoresymbol{TXF}{iint}
%\usepackage{wasysym}
\usepackage{amsthm}
\usepackage{mathrsfs}
\usepackage{txfonts}
\usepackage{stfloats}
\usepackage{cite}
\usepackage{cases}
\usepackage{subfig}
%\usepackage{hyperref}
%\usepackage{xtab}
\usepackage{longtable}
\usepackage{multirow}
%\usepackage{algorithm}
%\usepackage{algpseudocode}
\usepackage{enumitem}
\usepackage{mathtools}
\usepackage{iithtlc}
%\usepackage[framemethod=tikz]{mdframed}
\usepackage{listings}
\usepackage{listings}
    \usepackage[latin1]{inputenc}                                 %%
    \usepackage{color}                                            %%
    \usepackage{array}                                            %%
    \usepackage{longtable}                                        %%
    \usepackage{calc}                                             %%
    \usepackage{multirow}                                         %%
    \usepackage{hhline}                                           %%
    \usepackage{ifthen}                                           %%
  %optionally (for landscape tables embedded in another document): %%
    \usepackage{lscape}     



%\usepackage{stmaryrd}


%\usepackage{wasysym}
%\newcounter{MYtempeqncnt}
\DeclareMathOperator*{\Res}{Res}
%\renewcommand{\baselinestretch}{2}
\renewcommand\thesection{\arabic{section}}
\renewcommand\thesubsection{\thesection.\arabic{subsection}}
\renewcommand\thesubsubsection{\thesubsection.\arabic{subsubsection}}

\renewcommand\thesectiondis{\arabic{section}}
\renewcommand\thesubsectiondis{\thesectiondis.\arabic{subsection}}
\renewcommand\thesubsubsectiondis{\thesubsectiondis.\arabic{subsubsection}}

% correct bad hyphenation here
\hyphenation{op-tical net-works semi-conduc-tor}

%\lstset{
%language=C,
%frame=single, 
%breaklines=true
%}

%\lstset{
	%%basicstyle=\small\ttfamily\bfseries,
	%%numberstyle=\small\ttfamily,
	%language=Octave,
	%backgroundcolor=\color{white},
	%%frame=single,
	%%keywordstyle=\bfseries,
	%%breaklines=true,
	%%showstringspaces=false,
	%%xleftmargin=-10mm,
	%%aboveskip=-1mm,
	%%belowskip=0mm
%}

%\surroundwithmdframed[width=\columnwidth]{lstlisting}
\def\inputGnumericTable{}                                 %%
\lstset{
language=C,
frame=single, 
breaklines=true
}
 

\begin{document}
%

\theoremstyle{definition}
\newtheorem{theorem}{Theorem}[section]
\newtheorem{problem}{Problem}
\newtheorem{proposition}{Proposition}[section]
\newtheorem{lemma}{Lemma}[section]
\newtheorem{corollary}[theorem]{Corollary}
\newtheorem{example}{Example}[section]
\newtheorem{definition}{Definition}[section]
%\newtheorem{algorithm}{Algorithm}[section]
%\newtheorem{cor}{Corollary}
\newcommand{\BEQA}{\begin{eqnarray}}
\newcommand{\EEQA}{\end{eqnarray}}
\newcommand{\define}{\stackrel{\triangle}{=}}

\bibliographystyle{IEEEtran}
%\bibliographystyle{ieeetr}

\providecommand{\nCr}[2]{\,^{#1}C_{#2}} % nCr
\providecommand{\nPr}[2]{\,^{#1}P_{#2}} % nPr
\providecommand{\mbf}{\mathbf}
\providecommand{\pr}[1]{\ensuremath{\Pr\left(#1\right)}}
\providecommand{\qfunc}[1]{\ensuremath{Q\left(#1\right)}}
\providecommand{\sbrak}[1]{\ensuremath{{}\left[#1\right]}}
\providecommand{\lsbrak}[1]{\ensuremath{{}\left[#1\right.}}
\providecommand{\rsbrak}[1]{\ensuremath{{}\left.#1\right]}}
\providecommand{\brak}[1]{\ensuremath{\left(#1\right)}}
\providecommand{\lbrak}[1]{\ensuremath{\left(#1\right.}}
\providecommand{\rbrak}[1]{\ensuremath{\left.#1\right)}}
\providecommand{\cbrak}[1]{\ensuremath{\left\{#1\right\}}}
\providecommand{\lcbrak}[1]{\ensuremath{\left\{#1\right.}}
\providecommand{\rcbrak}[1]{\ensuremath{\left.#1\right\}}}
\theoremstyle{remark}
\newtheorem{rem}{Remark}
\newcommand{\sgn}{\mathop{\mathrm{sgn}}}
\providecommand{\abs}[1]{\left\vert#1\right\vert}
\providecommand{\res}[1]{\Res\displaylimits_{#1}} 
\providecommand{\norm}[1]{\lVert#1\rVert}
\providecommand{\mtx}[1]{\mathbf{#1}}
\providecommand{\mean}[1]{E\left[ #1 \right]}
\providecommand{\fourier}{\overset{\mathcal{F}}{ \rightleftharpoons}}
%\providecommand{\hilbert}{\overset{\mathcal{H}}{ \rightleftharpoons}}
\providecommand{\system}{\overset{\mathcal{H}}{ \longleftrightarrow}}
	%\newcommand{\solution}[2]{\textbf{Solution:}{#1}}
\newcommand{\solution}{\noindent \textbf{Solution: }}
\providecommand{\dec}[2]{\ensuremath{\overset{#1}{\underset{#2}{\gtrless}}}}
%\numberwithin{equation}{subsection}
\numberwithin{equation}{problem}
%\numberwithin{problem}{subsection}
%\numberwithin{definition}{subsection}
\makeatletter
\@addtoreset{figure}{problem}
\makeatother

\let\StandardTheFigure\thefigure
%\renewcommand{\thefigure}{\theproblem.\arabic{figure}}
\renewcommand{\thefigure}{\theproblem}


%\numberwithin{figure}{subsection}

%\numberwithin{equation}{subsection}
%\numberwithin{equation}{section}
%%\numberwithin{equation}{problem}
%%\numberwithin{problem}{subsection}
\numberwithin{problem}{section}
%%\numberwithin{definition}{subsection}
%\makeatletter
%\@addtoreset{figure}{problem}
%\makeatother
\makeatletter
\@addtoreset{table}{problem}
\makeatother

\let\StandardTheFigure\thefigure
\let\StandardTheTable\thetable
%%\renewcommand{\thefigure}{\theproblem.\arabic{figure}}
%\renewcommand{\thefigure}{\theproblem}
\renewcommand{\thetable}{\theproblem}
%%\numberwithin{figure}{section}

%%\numberwithin{figure}{subsection}



\def\putbox#1#2#3{\makebox[0in][l]{\makebox[#1][l]{}\raisebox{\baselineskip}[0in][0in]{\raisebox{#2}[0in][0in]{#3}}}}
     \def\rightbox#1{\makebox[0in][r]{#1}}
     \def\centbox#1{\makebox[0in]{#1}}
     \def\topbox#1{\raisebox{-\baselineskip}[0in][0in]{#1}}
     \def\midbox#1{\raisebox{-0.5\baselineskip}[0in][0in]{#1}}

\vspace{3cm}

\title{ 
	\logo{
Debian on Asus X205TA
	}
}



% paper title
% can use linebreaks \\ within to get better formatting as desired
%\title{Debian USB-Stick.}
%
%
% author names and IEEE memberships
% note positions of commas and nonbreaking spaces ( ~ ) LaTeX will not break
% a structure at a ~ so this keeps an author's name from being broken across
% two lines.
% use \thanks{} to gain access to the first footnote area
% a separate \thanks must be used for each paragraph as LaTeX2e's \thanks
% was not built to handle multiple paragraphs
%

\author{Sanjay Kumar, Tarun Chandrakar, Tanmay Agarwal and G V V Sharma$^{*}$% <-this % stops a space
\thanks{*The author is with the Department
of Electrical Engineering, Indian Institute of Technology, Hyderabad
502285 India e-mail:  gadepall@iith.ac.in. All content in this manual is released under GNU GPL.  Free and open source.}% <-this % stops a space
%\thanks{J. Doe and J. Doe are with Anonymous University.}% <-this % stops a space
%\thanks{Manuscript received April 19, 2005; revised January 11, 2007.}}
}
% note the % following the last \IEEEmembership and also \thanks - 
% these prevent an unwanted space from occurring between the last author name
% and the end of the author line. i.e., if you had this:
% 
% \author{....lastname \thanks{...} \thanks{...} }
%                     ^------------^------------^----Do not want these spaces!
%
% a space would be appended to the last name and could cause every name on that
% line to be shifted left slightly. This is one of those "LaTeX things". For
% instance, "\textbf{A} \textbf{B}" will typeset as "A B" not "AB". To get
% "AB" then you have to do: "\textbf{A}\textbf{B}"
% \thanks is no different in this regard, so shield the last } of each \thanks
% that ends a line with a % and do not let a space in before the next \thanks.
% Spaces after \IEEEmembership other than the last one are OK (and needed) as
% you are supposed to have spaces between the names. For what it is worth,
% this is a minor point as most people would not even notice if the said evil
% space somehow managed to creep in.



% The paper headers
%\markboth{Journal of \LaTeX\ Class Files,~Vol.~6, No.~1, January~2007}%
%{Shell \MakeLowercase{\textit{et al.}}: Bare Demo of IEEEtran.cls for Journals}
% The only time the second header will appear is for the odd numbered pages
% after the title page when using the twoside option.
% 
% *** Note that you probably will NOT want to include the author's ***
% *** name in the headers of peer review papers.                   ***
% You can use \ifCLASSOPTIONpeerreview for conditional compilation here if
% you desire.




% If you want to put a publisher's ID mark on the page you can do it like
% this:
%\IEEEpubid{0000--0000/00\$00.00~\copyright~2007 IEEE}
% Remember, if you use this you must call \IEEEpubidadjcol in the second
% column for its text to clear the IEEEpubid mark.



% make the title area
\maketitle

\tableofcontents

\bigskip

\begin{abstract}
%\boldmath
This manual lists the steps required to run Debian on an Asus X205TA EeeBook through a bootable pen drive.
\end{abstract}
% IEEEtran.cls defaults to using nonbold math in the Abstract.
% This preserves the distinction between vectors and scalars. However,
% if the journal you are submitting to favors bold math in the abstract,
% then you can use LaTeX's standard command \boldmath at the very start
% of the abstract to achieve this. Many IEEE journals frown on math
% in the abstract anyway.

% Note that keywords are not normally used for peerreview papers.
%\begin{IEEEkeywords}
%Cooperative diversity, decode and forward, piecewise linear
%\end{IEEEkeywords}



% For peer review papers, you can put extra information on the cover
% page as needed:
% \ifCLASSOPTIONpeerreview
% \begin{center} \bfseries EDICS Category: 3-BBND \end{center}
% \fi
%
% For peerreview papers, this IEEEtran command inserts a page break and
% creates the second title. It will be ignored for other modes.
\IEEEpeerreviewmaketitle


%\newpage
%\section{Component Pin Diagrams}
%%
%\input{chapter1}
%

%\newpage
\section{Resources}
\begin{enumerate}
\item Asus X205TA EeeBook 
  \item A bootable pen drive with an image file of Debian
  \item A blank pen drive with unallocated space(free space), min 8 GB size.
%  \item A Laptop with Secure Boot disabled.
\end{enumerate}


\subsection{Making a bootable pen drive}
We can make a pen bootable both by Linux as well as windows. For Linux we will be using disk-writer while for windows we will be using Rufus software.
\begin{enumerate}
\item First of all you have to download a image file of Debian from the following link
\begin{lstlisting}
https://cdimage.debian.org/debian-cd/current/i386/iso-cd/
\end{lstlisting}
\item Open the downloaded Image with disk-writer in case of Linux and with Rufus in case of Windows.
\item Now select the pen drive which you want to make bootable and click on restore.
\item It will take sometime. Once the process is finished the pen drive will be bootable.
\end{enumerate}

\subsection{Steps to Disable secure boot}
\begin{enumerate}
 
\item Go to bios setup. In order to go bios setup in X205TA EeeBook just press Esc button and boot setup will open.
\item Now Go to security Tag and select secure boot control and make it Disable.
\item Press F10 button to save and exit.
\item Secure boot is now disabled.

\end{enumerate}






%\input{./figs/components.tex}
\section{Steps for Installation  }
\subsection{\textbf{Installation}}

Refer to given link:
%\\
\begin{lstlisting}
http://www.elinuxbook.com/installation-of-linux-debian-9-stretch-with-snapshots/
\end{lstlisting}
\begin{enumerate}

\item Plug both the pen drive in the laptop.
\item Start the laptop and keep on pressing the esc button.
\item Select the pen drive that contain the image file of debian.
\item Click on Graphic Install.
\item Select the language. English
\item Next it will ask for location, just select India.
\item Now it will ask whether you want to install Hardware network for connected device. Just click on No and continue.
\item Select no Ethernet card from the next window and continue.
\item After this  configuration System will start to detect the CD-ROM for required files for further Installation of Linux Debian 9 Stretch.
\item Just Assign a Hostname for Debian 9 Operating System Then click on Continue.
\item It will ask you set the password for the root . just enter the password and click on continue.
\item Create an username. This name will used to login the Linux debian 9 system. Then Click on Continue.
\item Then set password for the User. 
\item Now we have to create Partitions for Linux Debian 9 Stretch Operating System. Here you will get multiple options to Create Partitions.Here we will create Partitions manually so select Manual. Click on Continue.
\item We have to make 2 partition. one for the boot and another for swap area.
\item Select the empty pen drive. it will ask that you have selected the whole disk do you want to make the partition. Just click on yes 
\item  Now select the free space, create new partition and give the space that you want for the boot section.
\item Now in partition setting select ext4 journalling
file system as Use as and select / as mount point
and select Done setting up the partition option.
\item Now it will be showing free unalloacated space.
\item Now similarly we will make a swap partition.In this in place of use as ext4 you have to click on swap area.
\item Now select Finish partitioning and write changes to disk and press enter.
\item Then complete the remaining installation.
\item It will ask for update package from mirror image. Just click on No and hit enter.
\item Now select required softwares for Linux Debian 9. Here we will be selecting all available Softwares and then click on Continue.
\item Now Linux debian 9 installation is Finishing
\item Now Restart your computer and remove only the bootable pen drive and boot the system from another pen drive by BIOS option.


\end{enumerate}
\section{Enabling Wifi} 
After the Installation their will be no wifi-driver. So in order to make wifi working just type the following commands in terminal.  These instructions are available at
\begin{lstlisting}
https://wiki.debian.org/InstallingDebianOn/Asus/X205TA
\end{lstlisting}

%\subsection{Steps foe making Wifi working}

\begin{enumerate}
    \item Connect your laptop to internet either by Lan or by USB tethering.
    
    \item Now open Terminal and type following command.
    \lstinputlisting{Command.txt}
    
    \item Reboot the system. Your wifi will be working.
\end{enumerate}

\section{Enabling Touchpad} 
After the installation configure the Touchpad. Follow the instructions below.

\begin{enumerate}

\item 
\begin{lstlisting}
su
cd /etc/X11
mkdir xorg.conf.d
\end{lstlisting}
\item Download 
\begin{lstlisting}
http://tlc.iith.ac.in/resources/sources.txt
\end{lstlisting}
\item Move the downloaded file to the appropriate directory with a new name.
\begin{lstlisting}
mv sources.txt /etc/X11/xorg.conf.d/40-libinput.conf 
\end{lstlisting}
%\subsection{Steps foe making Touchpad working}


%
%%\item 
%%
%%\item Just go to root and open the directory /etc/X11
%%\textbf{Command: su}
%%
%%\textbf{Command: cd /etc/X11}
%%\item Now make a directory inside X11 with command.
%%
%%\textbf{Command : mkdir xorg.conf.d}
%%\item 
%
%Now go to this directory and make a file.
%
%\textbf{command : nano 40-libinput.conf}
%
%\item 
%Now type the followings lines in the 
%    \lstinputlisting{Touchpad.txt}
%    
%\item Now just save the file by:-
%
%Press ctrl+x
%\\
%It will be ask if you want to save
%\\
%Press Y and hit Enter
%
\item Now Reboot your System

%\textbf{Command : Reboot}
\end{enumerate}

\section{Installing sudo}
In Debian, sudo command does not come by default. You have to install it.

%\subsection{Steps to Download to Sudo}
\begin{enumerate}
\item First goto root.
\item Download the file
\begin{lstlisting}
http://tlc.iith.ac.in/resources/Touchpad.txt
\end{lstlisting}
\item Move the downloaded file to the appropriate directory.
\begin{lstlisting}
mv Touchpad.txt /etc/apt/sources.list
\end{lstlisting}

%Then goto the Directory /etc/apt using cd command
%\item Open the sources.list file by
%\textbf{Command: nano sources.list}
%\item Replace the file with new file that is given below.
%\lstinputlisting{sources.txt}
%\item Now Save the File.
\item Install sudo
\begin{lstlisting}
apt-get update
apt-get install sudo
\end{lstlisting}
%Now download sudo and update
%\\
%\textbf{command : apt-get install sudo}
%\\
%\textbf{command : apt-get update}
\item Now you have to give Privileges to the user from the \textbf{sudoers} file.
\begin{lstlisting}
nano /etc/sudoers
\end{lstlisting}
% that will be in /etc directory
%\item Open the file With nano command.
\item In that file Under \textbf{user privilege specification} add one more line 
\begin{lstlisting}
username ALL=(ALL:ALL) ALL
\end{lstlisting}
\item Save and reboot.



\item sudo will work now.
\end{enumerate}

%\input{./chapters/chapter1}
%
%\section{Display Control through Arduino Software}
%\input{./chapters/chapter2}
%%
%\section{Decade Counter through Arduino}
%\input{./chapters/chapter3}
%%%
%\section{Karnaugh Maps}
%\input{./chapters/chapter4}
%%
%\section{Sequential Logic}
%\input{./chapters/chapter5}
%
%\section{C Programming}
%\input{./chapters/chapter6}

%\input{arduinoport}

%\bibliography{IEEEabrv,gvv_matrix}

%\input{chapter2} 
%%
%\newpage
%\section{$M$-ary Modulation}
%\input{chapter3} 
%
%\newpage
%\section{BER in Rayleigh Flat Slowly Fading Channels}
%\input{chapter4} 

\end{document}

