\documentclass[journal,12pt,onecolumn]{IEEEtran}
\usepackage{amsmath}
\usepackage{multicol}
\usepackage{enumerate}
\usepackage{amssymb}

\begin{document}
\providecommand{\nCr}[2]{\,^{#1}C_{#2}} % nCr
\providecommand{\nPr}[2]{\,^{#1}P_{#2}} % nPr
\providecommand{\mbf}{\mathbf}
\providecommand{\pr}[1]{\ensuremath{\Pr\left(#1\right)}}
\providecommand{\qfunc}[1]{\ensuremath{Q\left(#1\right)}}
\providecommand{\sbrak}[1]{\ensuremath{{}\left[#1\right]}}
\providecommand{\lsbrak}[1]{\ensuremath{{}\left[#1\right.}}
\providecommand{\rsbrak}[1]{\ensuremath{{}\left.#1\right]}}
\providecommand{\brak}[1]{\ensuremath{\left(#1\right)}}
\providecommand{\lbrak}[1]{\ensuremath{\left(#1\right.}}
\providecommand{\rbrak}[1]{\ensuremath{\left.#1\right)}}
\providecommand{\cbrak}[1]{\ensuremath{\left\{#1\right\}}}
\providecommand{\lcbrak}[1]{\ensuremath{\left\{#1\right.}}
\providecommand{\rcbrak}[1]{\ensuremath{\left.#1\right\}}}
\newcommand{\sgn}{\mathop{\mathrm{sgn}}}
\providecommand{\abs}[1]{\left\vert#1\right\vert}
\providecommand{\res}[1]{\Res\displaylimits_{#1}} 
\providecommand{\norm}[1]{\lVert#1\rVert}
\providecommand{\mtx}[1]{\mathbf{#1}}
\providecommand{\mean}[1]{E\left[ #1 \right]}
\providecommand{\fourier}{\overset{\mathcal{F}}{ \rightleftharpoons}}
\providecommand{\hilbert}{\overset{\mathcal{H}}{ \rightleftharpoons}}
\providecommand{\system}{\overset{\mathcal{H}}{ \longleftrightarrow}}

\newcommand{\solution}{\noindent \textbf{Solution: }}
\providecommand{\dec}[2]{\ensuremath{\overset{#1}{\underset{#2}{\gtrless}}}}
\title{ 
Problem Set: Calculus
}
\author{J.~Balasubramaniam$^{\dagger}$ %<-this  stops a space
\thanks{$\dagger$ The author is with the Department of Mathematics, IIT Hyderabad
502285 India e-mail: jbala@iith.ac.in. }
}
\maketitle
\section{Continuity}
\begin{enumerate}
 
\item Show that the following functions are continuous by using $\epsilon - \delta$ definition.

\begin{enumerate}[(i)]
\begin{multicols}{4}
\setlength\itemsep{2em}
\item $
x^n
$
\item $
\sin x
$
\item $
e^{2x}
$
\item $
\ln x
$
\item $
\sqrt{x}
$
\item $
\sin ^{-1}(x)
$
\item $
x^2+x
$
\item $
\frac{x+1}{x-2} , x \neq 2 
$
\item $
\sinh 4x
$
\item $
\ln(\sin x)
$
\item $
\sin (2x) + x^2
$
\item $
{(x+2)^3}
$

\end{multicols}
\end{enumerate}

\item Investigate the continuity of each of the following functions at the indicated point $x_{0}$.
\begin{enumerate}[(i)]
\begin{multicols}{2}


\item $
f(x)=\begin{cases}
\frac{\sin x}{x},& x\neq0 ;\\
0,&x=0
\end{cases}
,x_{0}=0.
$

\item $
f(x)=x-|x|,x_{0}=0.
$

\item $
f(x)=\begin{cases}
\frac{x^3-8}{x^2-4},& x\neq2 ;\\
3,&x=2
\end{cases}
,x_{0}=2.
$

\item $
f(x)=\begin{cases}
\sin \pi x,& 0<x<1 \\
\ln x,& 1<x<2
\end{cases}
,x_{0}=1.
$



\end{multicols}
\end{enumerate}


\item Give examples of functions with the following properties:
\setlength\itemsep{2em}
\begin{enumerate}[(a)]

\item A function $f$ which is continuous at \textit{only} a finite number of points.
\item A function $f$ which is discontinuous at \textit{only} a finite number of points.
\item An $f$ which is nowhere continuous on the real line.
\item An $f$ which is continuous at every rational number.
\item An $f$ which is discontinuous at every rational and continuous at every irrational on $(0,\infty)$.
\item A $c \in \mathbb{R}$ and two functions $f, g$ that are discontinuous at $c$ but such that $f+g$ and $fg$ are continuous at $c$.
\end{enumerate}

\item Prove that any polynomial of finite degree over $\mathbb{R}$ is continuous.Hence show that rational functions are continuous over their admissible domain.

\item Let $f$,$g$ be two continuous functions on $(a,b)$ such that $f(x)=g(x)$ for every rational $x\in (a,b)$.Prove that $f(x)=g(x)$ for all $x \in (a,b)$.

\item Determine the domain of continuity of the following functions.

\begin{enumerate}[(i)]
\begin{multicols}{3}
\setlength\itemsep{2em}
\item $
\sqrt{1-x^2}
$
\item $
\sin (e^{-x^2})
$
\item $
\ln(1+\sin x)
$
\item $
\frac{1+\cos x}{3+\sin x}
$
\item $
\frac{1}{\sqrt{10+x}}
$
\item $
\sin \frac{1}{(x-1)^2}
$
\item $
\sin \brak{\frac{1}{\cos x}}
$
\item $
\sqrt{(x-3)(6-x)} 
$
\item $
\begin{cases}
x^2 \sin {\brak{\frac{1}{x}}},& x\neq0 \\
0,&x=0
\end{cases}
$

\item $
\sqrt{{x+\sqrt{x}}}
$
\item $
\cos (\sqrt{1+x^2})
$
\item $
\frac{\sqrt{1+\abs\sin x}}{x}
$
\end{multicols}
\end{enumerate}


\item
\setlength\itemsep{2em}
\begin{enumerate}[(a)]

\item Prove that 
 $
f(x)=\begin{cases}
x \sin \brak{\frac{1}{x}},& x\neq0 \\
5,&x=0
\end{cases}
,$
is not continuous at $x=0$. 
\item Can you redefine $f(0)$ so that $f$ is continuous at $x=0$.
\end{enumerate}

\item Let $f$ : $\mathbb{R}$ $\rightarrow$ $\mathbb{R}$ be continuous on $\mathbb{R}$.

\setlength\itemsep{2em}
\begin{enumerate}[(a)]

\item If $f(r)=0$ for every rational number $r$ , then prove that $f(x)=0$ for all $x\in\mathbb{R}$.
\item If $f(r)=0$ for every irrational number $r$ , then is it true that $f(x)=0$ on whole of $\mathbb{R}$.
\end{enumerate}

\item Show that every polynomial of odd degree with real coefficients has atleast one real root.

\item
Prove that 
$\lim\limits_{x \to 0}{\frac{x^2\sin\brak{\frac{1}{x}}}{\sin x}}=0$.

\item Let $I$ = $[a,b]$ and $f: I \rightarrow \mathbb{R}$ be a continuous function.

\setlength\itemsep{2em}
\begin{enumerate}[(a)]
\item If $f(x)>0$ for all $x$ $\in$ $I$ ,show that there exists an $\alpha>0$ such that $f(x)\geqslant\alpha$ for all $x$ $\in$ $I$.
\item If for each $x$ $\in$ $I$ there exists a $y$ $\in$ $I$ such that $\abs{f(y)}\leqslant\frac{1}{2}f(x)$,then show that there exists a $c$ $\in$ $I$ such that $f(c)=0$.
\item Show that $f$ is bounded on $I$, i.e., there exists an $M>0$ such that $\abs{f(x)} < M$  for all $x$ $\in$ $I$. 
What if the interval $I=(-\infty,b]$ or $I=[a,\infty)$,will $f$ still be bounded? 
\item Show that $f$ attains both maximum and minimum values over $I$, i.e., there exists $p,q \in I$ such that $f(p)$ $\leqslant$ $f(x)$ $\leqslant$ $f(q)$ for all $x$ $\in$ $I$.What happens if either $a=-\infty$ or $b=\infty$? 
\item Let $I$ =$[0,1]$ and $f(0)=f(1)$. Prove that there exists a point $c$ $\in$ $[0,\frac{1}{2}]$ such that $f(c)=f\brak{c+\frac{1}{2}}$.
\item Let $I$ =$[0,1]$ and let $0$ $\leqslant$ $f(x)$ $\leqslant$ $1$ for all $x$ $\in$ $I$. Show that there exists a point $c$ $\in$ $I$ such that $f(c)=c$, i.e., $f$ has a fixed point.
\item Show that if $f(a)$ $\leqslant$ $a$ and $f(b)$ $\geqslant$ $b$ then $f$ has a fixed point in $I$, i.e., there exists a point $c$ $\in$ $I$ such that $f(c)=c$.
\end{enumerate}


\item Let $K>0$ and let $f:\mathbb{R} \rightarrow \mathbb{R}$ satisfy the condition $\abs{f(x)-f(y)}$ $\leqslant$ $K\abs{x-y}$ for all $x,y\in\mathbb{R}$. Show that $f$ is continuous at every point $c\in\mathbb{R}$.

\item Define $g$ : $\mathbb{R}\rightarrow\mathbb{R}$ as follows.
$
g(x)=\begin{cases}
2x,& x\in\mathbb{Q} \\
x+3,& x\in\mathbb{R} \setminus \mathbb{Q}
\end{cases}
.$
Find all the points at which $g$ is continuous.

\item Let $A$ $\subseteq$ $B$ $\subseteq$ $\mathbb{R}$, $f:\mathbb{B} \rightarrow \mathbb{R}$ and $g$ be the restriction of $f$ to $A$, i.e., $g(x)=f(x)$ for $x\in A$.

\setlength\itemsep{2em}
\begin{itemize}
\item If $f$ is continuous at some $c\in A$ , then show that $g$ is also continuous at $c$.
\item Show by example that $g$ is continuous at some $c\in A$ does not necessarily imply that $f$ is continuous at that $c$.
\end{itemize}

\item Let $A$, $B$ $\subseteq$ $\mathbb{R}$, $f:A \rightarrow B$ and $g:B \rightarrow \mathbb{R}$. If $f$ is continuous on $A$ and $g$ is continuous on $B$, show that $g\circ f$  is continuous on $f(A)$.

\item While the intermediate value theorem assures you of a root of a continuous function which assumes both negative and positive values on a bounded interval,how do you actually find the roots? Read up on some such methods, viz., Bisection Method,Newton-Raphson method, etc. Investigate the additional requirements on $f$ for such methods to be applicable.

\item Let $f:[a,b]\rightarrow\mathbb{R}$ be integrable on $[a,x]$ for every $x\in [a,b]$ and let the indefinite integral $A(x)$ be defined as
\begin{equation*}
A(x)=\int_{a}^{x}f(t)dt.\\
\end{equation*}
Show that $A$ is continuous on whole of $[a,b]$.(At each end point we have one-sided continuity.)\\

\section{Differentiation}


\item Find the derivatives of the following functions from the definition.                   

\begin{enumerate}[(i)]
\begin{multicols}{3}
\setlength\itemsep{2em}
\item $
\frac{3+x}{3-x},x\neq 3
$
\item $
\sqrt{2x-1}
$
\item $
\ln(1+\sin x)
$
\item $
\frac{1+\cos x}{3+\sin x}
$
\item $
\frac{1}{\sqrt{10+x}}
$
\item $
\sin \frac{1}{(x-1)^2}
$
\item $
\sin \brak{\frac{1}{\cos x}}
$
\item 
$
\!
\begin{aligned}[t]
\sqrt{(x-3)(6-x)}, 
\\
3\leqslant x \leqslant 6 
\end{aligned}
$ 
\item $
x^2 \sin \brak{\frac{1}{x}}, x\neq 0 ;f(0)=0
$

\item $
\sqrt{{x+\sqrt{x}}}
$
\item 
$
\cos (\sqrt{1+x^2})
$
\item 
$
\frac{\sqrt{1+\abs{\sin x}}}{x}
$
\end{multicols}
\end{enumerate}

\item Let 
 $
f(x)=\begin{cases}
x^2 \sin \frac{1}{x},& x\neq0 \\
0,&x=0.
\end{cases}
$


\setlength\itemsep{2em}
\begin{enumerate}[(a).]

\item Is $f$ differentiable at $x=0$?
\item Is $f'$ is continuous at $x=0$?
\end{enumerate}

\item Use L'Hospital's rule to evaluate the following limits.

\begin{enumerate}[(i)]
\begin{multicols}{3}
\setlength\itemsep{2em}
\item
$\lim\limits_{x \to 0}{\frac{e^{2x}-1}{x}}$

\item
$\lim\limits_{x \to 0}{\frac{{1+\cos \pi x}}{x^2-2x+1}}$

\item
$\lim\limits_{x \to \infty}{\frac{{3x^2-x+5}}{5x^2+6x-3}}$

\item
$\lim\limits_{x \to \infty}{{x^2}}{e^{-x}}$

\item
$\lim\limits_{x \to 0}{(\cos x) ^ \frac{1}{x^2}}$

\item
$\lim\limits_{x \to 1}{\frac{(2x-x^4)^\frac{1}{2}-x^\frac{1}{3}}{1-x^\frac{3}{4}}}$

\item
$\lim\limits_{x \to \infty}{\frac{{5x^2-3x}}{7x^2+1}}$

\item
$\lim\limits_{x \to \infty}\brak{x-\sqrt{x+x^2}}$

\item
$\lim\limits_{x \to \infty}{\frac{\sqrt{x+2}}{\sqrt{x+1}}}$


\end{multicols}
\end{enumerate}

\end{enumerate}
\end{document}



